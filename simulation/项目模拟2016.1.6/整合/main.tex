\documentclass{article}
\usepackage{CJK,graphicx}
\usepackage{amsmath,geometry}
\geometry{left=2.5cm,right=2.5cm,top=2.5cm,bottom=2.5cm}
\begin{document}
	\begin{CJK}{UTF8}{gbsn}
		\title{\bf “科大·温馨家园”代培生系列活动\\总策划书}
		\author{江金健, 靳亚雪,周倩芳,王怡然,张宇,苏鹏,王叶竹,许仕杰}
		\date{Jan. 7, 2017}
				\begin{figure}
					\centering
					\includegraphics[width=0.5\textwidth]{logo-wb.jpg}
				\end{figure}
		\maketitle
		%封面不设页码
		\setcounter{page}{0}
		\thispagestyle{empty}
		%
		\newpage
		\tableofcontents
		
		\newpage
		\section{研究生篮球赛策划书}
		\subsection{活动简介}
		代培生篮球赛是以科大广大代培生为主题的大型活动,旨在加强科大代培生集体的文化和环境建设,促进代培生更好更快地融入到科大这个温馨的大家庭,同时也丰富研究生同学的课余文化生活,让同学们加强户外体育锻炼,增进同学之间的感情,给同学们打造一个相互交流、相互学习的平台。
		
		本次比赛分为两个部分:一是趣味赛,趣味赛要求男女混合组队(一共四名队员,女生至少一名)。二是正赛,代培生以所为单位进行报名,组队后进行淘汰比赛。
		
		\subsection{项目组成员联系方式}
		
		\begin{table}[htbp]
			\centering
			\begin{tabular}{|c|c|c|}
				\hline
				姓名 & 联系方式 & 分工\\\hline
				& & \\\hline
				& & \\\hline
				& & \\\hline
				& & \\\hline
				& & \\\hline
				& & \\\hline
				& & \\\hline
			\end{tabular}
		\end{table}
		
		\subsection{趣味篮球赛比赛项目}
		\begin{enumerate}
			\item 蛙跳+运球上篮,先完成蛙跳,然后每人依次上篮。(上篮成功两次即过)\\
			蛙跳规则:
			\begin{enumerate}
				\item 蛙跳距离:篮球场的宽度
				\item 蛙跳方式:
				\begin{enumerate}
					\item 1人从篮球场一侧跳到另一侧,然后再返回原点。
					\item 2人从篮球场一侧跳到另一侧,然后再返回原点。后面的人必须拉住前面的人的衣服,负责算违规。
					\item 3人从篮球场一侧跳到另一侧,然后再返回原点。后面的人必须拉住前面的人的衣服,否则算违规。
					\item 4人从篮球场一侧跳到另一侧,然后再返回原点。后面的人必须拉住前面的人的衣服,负责算违规。
				\end{enumerate}
			\end{enumerate}
							\item 保龄篮球:\\
							规则:参赛队员原地转圈十次,转圈完成后用手中的篮球滚击圈中的矿泉水瓶,击中5个矿泉水瓶获胜。每队需要击中两次。
							\item 男女搭配,投篮不累\\
							规则:4名队员分成两组,每组队员用各自的气球拱起,从中圈出发,运送到底线,然后一人站在三分线外,另一人站在禁区弧线内,三分线外的组员负责投篮,弧线内的组员用箱子接球,每组五次投篮机会,投中结束。比赛过程中,不得用手或者身体其他部位触碰到气球,否则视为违规。
							\item 注意事项:不要将气球弄爆,气球准备三个即可,避免不必要的清扫工作。
							\item 场地大小:半场
		\end{enumerate}
		每完成一个项目,将获得一张卡片。集齐四张卡片饥渴到兑奖处兑奖。按完成项目时间的先后顺序进行奖品的分发。
		
		\subsection{趣味篮球赛注意事项}
		\begin{enumerate}
			\item 趣味篮球赛的比赛时间初步定在10月16日(周六),视天气情况而定。
			\item 所有工作人员统一着装,预计万门大学的文化衫
			\item 每个半场将有两名工作人员,人数不够的话,拉外援,做好分工工作,一名负责讲解规则、判断完成情况、发放完成的卡片,另一名负责协助。
			\item 每队同一时间段只能参加一个项目,由队长负责排队,队长身上将贴上标签,在等候区进行等候。
			\item 活动当天现场工作人员分工:
			\begin{enumerate}
				\item 兑奖区:1名,负责发放奖品
				\item 规则讲解员: 1名,负责讲解规则
				\item 每个半场:2名,共六个半场,负责活动顺利进行
				\item 摄影:1名,负责拍摄,活动未开始之前来张合照。
			\end{enumerate}
			工作人员合计15人。
		\end{enumerate}
		
		\subsection{正赛}
		正赛视报名队伍进行分组,赛制为淘汰赛,时间为10月下旬,具体时间将与队长协商。
		\subsection{篮球赛总体日程安排}
		趣味赛:
		\begin{itemize}
			\item 9月28日	:项目组启动会,明确分工,项目启动。
			\item 10月1日~10月5日:完成海报、横幅的制作与宣传稿的撰写
			\item 10月6日~10月8日:海报的张贴、申请场地、申请借球等
			\item 10月9日~10月15日:进行报名统计工作。
			\item 10月16日:活动进行。
		\end{itemize}
		正赛:\\
		正赛在趣味赛的后一周周末举行,淘汰制。
		
		\subsection{经费预算}
		\begin{table}[htbp]
			\centering
			\begin{tabular}{|c|c|c|}
				\hline
				项目 & 价格 & 备注 \\\hline
				篮球 & 150*2=300 & 可从体教中心借\\\hline
				记分牌	&30*2=60&	同上\\\hline
				哨&	5*5=25	&同上\\\hline
				秒表&	30*2=60&	同上\\\hline
				横幅&	100	&\\\hline
				海报及展架&	200&	\\\hline
				奖杯&	50&	具体数额不定, \\\hline
				奖牌	&15*36=540&	同上\\\hline
				奖品&	1320&	同上\\\hline
				比赛用水&	120	&\\\hline
				合计&	2775&	\\\hline
			\end{tabular}
		\end{table}
		\subsection{应急预案}
		\begin{itemize}
			\item 			如果在比赛中途遇到下雨,则启用室内备用场地。
			
			\item 如不能使用备用场地,裁判应立即中止比赛,场监记录当时比赛进行的时间、比分,将情况如实填写并由双方队长签字确认;
			
		\item 	如果比赛进行时间不足5分钟,则之前比分无效,由组委会安排重赛
			;
		\item 	如果比赛剩余时间不足5分钟,则由落后方选择补赛或保留当时比分为最后结果,若平局,则安排补赛;
			
		\item 	其余情况由组委会与双方队员协商安排补赛;
			
		\item 	补赛时保留原有比分、时间、犯规记录,补赛的中场休息由当值裁判视情况决定;
			
		\item 	其他类似情况造成的比赛中断适用此办法。
		\end{itemize}
		
%%%%%%%%%%%%%%%%%%%%%%%%%%%%%%%%%%%%%%%%%%%%%%%%%%%%%%%%%%%%%%%%%%%%%

	\section{羽毛球友谊赛}
	\subsection{活动概况}
		活动主办方:中科大研究生协会\\
		活动时间:待定\\
		活动地点:中科大东校区\\
	\subsection{活动目的}
通过羽毛球比赛来丰富大家的业余生活,强身健体,增进代培生相互之间的感情,促进各大院系之间的友谊,加强代培生与本校学生的交流。

\subsection{活动宗旨}
本次比赛本着促进代培生之间以及与本校学生之间的交流,加强各院系之间合作的目的,坚持友谊第一,比赛第二的原则。

\subsection{前期准备}
\subsubsection{宣传工作:}
\begin{enumerate}
	\item 海报制作:2张喷绘海报(东区篮球场+美食广场)
	\item 横幅:申请并联系制作横幅(内容为“中科大温馨家园代培生系列活动-羽毛球友谊赛”)
\end{enumerate}

\subsubsection{联系工作:}
\begin{enumerate}
	\item 联系羽毛球协会:联系羽毛球协会并邀请羽毛球协会派出5名裁判参加本次友谊赛,担任裁判。
	\item 通知各院系导员:通知各院系班级派出代培生一男一女,非代培生一男一女参加本次羽毛球友谊赛。
\end{enumerate}

\subsubsection{汇总分组工作:}汇总参加比赛选手名单,并进行友谊赛分组
\subsubsection{分组结束后的通知工作:}通知所有组织及相应知道老师本次比赛具体安排,发放秩序册。
\subsubsection{场地租借及物资准备:}具体安排见附录

\subsection{活动流程}
本次羽毛球友谊赛采取11分制,每场比赛只进行一局,即先获得11分的队伍获胜,若打至10平,则需净胜2分以上方可取得比赛胜利(打至14平时,先取得第15分的一方获胜)。每场比赛预计15分钟且为混合双打的形式。

全程比赛按照小组赛、淘汰赛、半决赛和决赛的顺序进行,决出一、二、三等奖。

\subsection{后期工作}
\begin{itemize}
\item 新闻稿撰写
\item 物资归还整理
\item 场地清理
\end{itemize}


\subsection{经费预算}
		\begin{table}[htbp]
			\centering
			\begin{tabular}{|c|c|c|c|}
				\hline

项目	&单价	&总量&	小记\\\hline
水	&3/桶&	5&	15\\\hline
横幅&	120	&1&	120\\\hline
气球&	8/袋&	1&	8\\\hline
器材	&5/套&	50&	250\\\hline
奖品	&	&	&60\\\hline
总计&		&&	603\\\hline

\end{tabular}
\end{table}

\subsection{附 录}
\subsubsection{人员名单}
\begin{table}[htbp]
	\centering
	\begin{tabular}{|c|c|}
		\hline	
总负责	& \hspace{20em} \\\hline
现场秩序维护	& \\\hline
计分人员	& \\\hline
统计人员	& \\\hline
裁判员	& \\\hline
主持人	& \\\hline
\end{tabular}
\end{table}

\subsubsection{分组名单}

\begin{table}[htbp]
	A组:
	\centering
	\begin{tabular}{|c|c|c|}
		\hline	
		序号& 姓名(1男1女)& 	所属组织\\\hline
		A1	&&\\\hline
		A2	&&	\\\hline
		A3	&&	\\\hline
		A4	&&\\\hline
	\end{tabular}
\end{table}


\begin{table}[htbp]
	B组:\centering
	\begin{tabular}{|c|c|c|}
		\hline	
		序号& 姓名(1男1女)& 	所属组织\\\hline
		B1	&&\\\hline
		B2	&&	\\\hline
		B3	&&	\\\hline
		B4	&&\\\hline
	\end{tabular}
\end{table}


\begin{table}[htbp]
	C组:\centering
	\begin{tabular}{|c|c|c|}
		\hline	
		序号& 姓名(1男1女)& 	所属组织\\\hline
		C1	&&\\\hline
		C2	&&	\\\hline
		C3	&&	\\\hline
		C4	&&\\\hline
	\end{tabular}
\end{table}


\begin{table}[htbp]
D组:	\centering
	\begin{tabular}{|c|c|c|}
		\hline	
		序号& 姓名(1男1女)& 	所属组织\\\hline
		D1	&&\\\hline
		D2	&&	\\\hline
		D3	&&	\\\hline
		D4	&&\\\hline
	\end{tabular}
\end{table}


\begin{table}[htbp]
E组:	\centering
	\begin{tabular}{|c|c|c|}
		\hline	
		序号& 姓名(1男1女)& 	所属组织\\\hline
		E1	&&\\\hline
		E2	&&	\\\hline
		E3	&&	\\\hline
		E4	&&\\\hline
	\end{tabular}
\end{table}


\begin{table}[htbp]
F组:	\centering
	\begin{tabular}{|c|c|c|}
		\hline	
		序号& 姓名(1男1女)& 	所属组织\\\hline
		F1	&&\\\hline
		F2	&&	\\\hline
		F3	&&	\\\hline
		F4	&&\\\hline
	\end{tabular}
\end{table}


\begin{table}[htbp]
G组:	\centering
	\begin{tabular}{|c|c|c|}
		\hline	
		序号& 姓名(1男1女)& 	所属组织\\\hline
		G1	&&\\\hline
		G2	&&	\\\hline
		G3	&&	\\\hline
		G4	&&\\\hline
	\end{tabular}
\end{table}


\begin{table}[htbp]
H组:	\centering
	\begin{tabular}{|c|c|c|}
		\hline	
		序号& 姓名(1男1女)& 	所属组织\\\hline
		H1	&&\\\hline
		H2	&&	\\\hline
		H3	&&	\\\hline
		H4	&&\\\hline
	\end{tabular}
\end{table}


\begin{table}[htbp]
I组:	\centering
	\begin{tabular}{|c|c|c|}
		\hline	
		序号& 姓名(1男1女)& 	所属组织\\\hline
		I1	&&\\\hline
		I2	&&	\\\hline
		I3	&&	\\\hline
		I4	&&\\\hline
	\end{tabular}
\end{table}


\begin{table}[htbp]
J组:	\centering
	\begin{tabular}{|c|c|c|}
		\hline	
		序号& 姓名(1男1女)& 	所属组织\\\hline
		J1	&&\\\hline
		J2	&&	\\\hline
		J3	&&	\\\hline
		J4	&&\\\hline
	\end{tabular}
\end{table}


\begin{table}[htbp]
K组:	\centering
	\begin{tabular}{|c|c|c|}
		\hline	
		序号& 姓名(1男1女)& 	所属组织\\\hline
		K1	&&\\\hline
		K2	&&	\\\hline
		K3	&&	\\\hline
		K4	&&\\\hline
	\end{tabular}
\end{table}


\begin{table}[htbp]
L组:	\centering
	\begin{tabular}{|c|c|c|}
		\hline	
		序号& 姓名(1男1女)& 	所属组织\\\hline
		L1	&&\\\hline
		L2	&&	\\\hline
		L3	&&	\\\hline
		L4	&&\\\hline
	\end{tabular}
\end{table}

\subsubsection{比赛时间}
\begin{itemize}
	\item 13:20-13:40小组赛一(进行A到D组比赛,每组1号组合对阵2号组合,3号组合对阵4号组合,八场比赛同时进行)
	\item 13:40-14:00小组赛二(进行E到H组比赛,每组1号组合对阵2号组合,3号组合对阵4号组合,八场比赛同时进行)
	\item 14:00-14:20小组赛三(进行I到L组比赛,每组1号组合对阵2号组合,3号组合对阵4号组合,八场比赛同时进行)
	\item 14:00-14:20淘汰赛一(进行A到H组淘汰赛,每组小组赛胜出的两对组合直接对决,八场比赛同时进行)
	\item 14:20-14:40淘汰赛二(进行I到L组淘汰赛,每组小组赛胜出的两对组合直接对决,四场比赛同时进行)
	\item 15:00-15:20半决赛(每小组胜出组合按照A组对B组,C组对D组,E组对F组,G组对H组,I组对J组,K组对L组的比赛形式同时进行比赛)
	\item 15:35-16:00决赛(半决赛每场比赛胜出组合进行决赛,三场决赛同时进行,A组对B组优胜者对C组对D组优胜者,E组对F组优胜者对G组对H组优胜者,	I组对J组优胜者对K组对L组优胜者)
\end{itemize}



\subsubsection{比赛奖项}
一等奖:决赛3个优胜组合\\
二等奖:决赛3个落败组合\\
三等奖:半决赛6个落败组合\\

%%%%%%%%%%%%%%%%%%%%%%%%%%%%%%%%%%%%%%%%%%%%%%%%%%%%%%%%%%%%%%%%%%%%%


\section{素质拓展活动简单策划}
\subsection{活动目的:}
进一步加强代培生同学间的沟通与交流, 友谊合作,拼搏进取

\subsection{活动对象}
科大代培生同学

\subsection{活动地点与大致时间:}
以前期宣传后报名人数为准,人数过多则在操场举办,否则申请西活多功能厅。选择一个晴朗的周末的下午举办。

项目启动与策划分工:1天

前期宣传与等待同学报名:1周

申请场地与报名统计等工作:2天

\subsection{活动具体环节:}
\begin{enumerate}
	\item 首先进行热身游戏(a)
	
	\item 素质拓展过程
	\begin{itemize}
			\item Case1:人数不超过100,那么可以进行团队竞争类游戏,‘获胜’队伍可获奖品(科大纪念品等)
			\begin{enumerate}
					\item	根据人数分4队,每队大约10~25人(注意合理男女搭配,并且尽量跨所组队),选出一名队长,请队长发表就职演说
					\item	确定队名与队徽、口号,分队演示,后以队名称呼某队,积分制进行游戏,每个游戏第一名4分,第二名3分,第三名2分,第四名1分
					\item	首先进行队内小游戏(b),以相互认识队员,再依次进行游戏①②(团体竞赛游戏即可)
					\item 游戏结束后四个队长石头剪刀布(哈哈哈),决出最后的‘获胜队伍’		
			\end{enumerate}

			\item Case2:人数超过100,那么分区域游戏,可以设计个人积分制度,玩一个游戏积1分,游戏结束后可用积分兑换奖品
			
			根据人数分成数个区域,每个区域大约10~25人,所有同学自由参加,由工作人员担任每个游戏区域的负责人,按区域进行游戏,同学可自由组队参加某个区域的游戏,但必修要有女生。
			
			开始时首先进行区域内小游戏(b),以相互认识队员,游戏进行一段时间后取消此预备游戏,再依区域进行相应游戏(趣味性团体游戏即可),游戏至活动结束时间
	\end{itemize}

	
	\item  活动结束后,发放奖品,邀请校领导或老师进行致辞,表达对代培生同学的关心与热情,致结束辞
\end{enumerate}


\subsection{人员分工}
主持人:1~2人\\
拍摄:1人\\
协助游戏人员:至少4人(每组/区域需有一名协助/主持游戏人员)\\

\subsection{经费大致预算}
横幅+海报+奖品,******(我也不知道大概要多少钱)元。奖品可为科大纪念品,如书签、明信片等(人少可适当添加稍贵点的物品,人多就算了)

室内活动还有矿泉水

合计:我也不造要多少钱QAQ  不过没什么器材花钱

\subsection{注意事项}
\begin{enumerate}
	\item	保持严格的组织性和纪律性,各组成员需听从指挥,不得擅自行动
	\item	协助游戏人员负责对必要的活动场地进行布置并对本次拓展活动进行监督管理,判定活动项目的完成与否,并对项目完成时间进行记录 
	\item	队长对本队伍负责,必要时需组织本队队员喊口号,唱队歌鼓舞本队队员士气并时刻注意队员的健康状况与安全问题
\item	活动中队员不得擅自脱离队伍,行进中禁止打闹
	\item	活动过程中禁止乱扔垃圾,保护环境,爱护公共设施
\end{enumerate}


\subsection{后期工作}
\begin{enumerate}
\item 在每项活动举办后对所用经费进行统计报销。
\item 活动中的场景拍摄得到的照片以及活动记录制作简报,保存纪念。
\end{enumerate}





\subsection{附:游戏项目举例}
\begin{enumerate}
	

\item 你说我做:全体同学排列好,由一名工作人员担任教练,面向同学们,喊出“教练说+动作”,然后同学们执行该动作,例如教练喊出“教练上举起右手”,那么同学们便举起右手,但如果教练说的话没有“教练说”三个字,则视为无效,同学们不可以做动作。例如教练喊出“举起右手”,那么不应该做动作。做错动作或不做相应动作的同学走到教练身后。最终留到最后的同学获得简单奖品
\item 友情链接:各队成员任意派成一列,由排在首位者向他/她背后的人作自我介绍,如:我是数学院计算数学专业的**,第二名队员再对他/她背后的做介绍,如:我是数学院计算数学专业的**后边的***,第三名队员则依此介绍说:我是数学院计算数学专业的xx后边的***后边的***,依此类推
\item 福尔摩斯:每队随机抽出两名队员,让每人仔细观察自己的搭档1分钟,一分钟后,两人转过脸去,背对着站立。这时,由别队成员分别向他们二人提出有关自己搭档的体貌特征的五个问题。以每队答出问题多少决定积分
\item 潮起潮落:每队人全部上场一人做旅行者,其他把人分两排面对面站着,队首两人站在终点前队尾两人将旅行者举过头顶是旅行者平躺在四人举起的手上,然后队尾两人跑到队首旅行者就往前传一点,然后后面的两个人在跑道队首一点点把旅行者举着往前移,直到终点。记下每队所用时间
\item 大风吹:大家围坐一圈,中间站一个人。此人说大风吹,众人问“吹什么”,此人回答吹一样东西,满足该东西的人就必须交换位置,不得和两边的人交换位置,此人参与抢位置的过程,未抢到位置的人站在中间继续说。比如此人可说“吹戴首饰的人”,那么戴首饰的人就必须交换位置。
\item 更多小游戏在此不一一列举……
\end{enumerate}
%%%%%%%%%%%%%%%%%%%%%%%%%%%%%%%%%%%%%%%%%%%%%%%%%%%%%%%%%%%%%%%%%%%%%
\section{外联赞助部分}

\subsection{商家:}
\begin{enumerate}
	\item 饮料类
	农夫山泉:现场矿泉水需要较多\\
	红牛运动功能性饮料
	\item 运动服饰类:
	NIKI(比赛的服装)、红双喜(比赛需要的球类和球拍等)除了物品的支持,这些大的商家希望通过联系,可以提供一定的经费支持。
	\item 运动装备类:东区这边有许多的售卖户外运动装备的,以及一些健身器材的,包括百脑汇那边的健身房,都可以联系,通过帮助其宣传,希望可以提供一定的经费支持。
	\item 学校周边的药店类:学校周边的希望可以提供云南白药、创口贴、双氧水、葡萄糖等一些药品。
	\item 学校周边的食品店:因为学生群体订外卖较多,可以考虑联系附近的食品类商家,为我们提供一定的经费支持。
	\item 学校周边的电子产品店:学校周边的电脑电子产品店铺较多,可以提供一些奖品支持。这部分商家数量多,其之间竞争力也较大。
	备注:除了联系一些大型的品牌商家,也要充分利用好学校周边的资源。因为利益相关较为密切,可能更容易得到相关的经费支持。
\end{enumerate}
\subsection{联系的社团:(主要提供人员支持)}
\begin{enumerate}
	\item 学生篮球协会:比较专业,可以提供比赛所需要的球,提供现场裁判人员。并且在运动伤害产生的时候,可以即时帮忙处理。
	\item 学生羽毛球协会:比较专业,提供比赛的裁判和现场维护人员。比赛的球拍以及羽毛球。在运动伤害从产生时,帮忙及时处理。
	\item 学生舞蹈团:活动开场和中间休息的放松类节目,出一个啦啦队,在比赛活动的时候加油。
	\item  中国科学技术大学全媒体中心:除了我们自己的公众号平台,还可以和全媒体中心联合,利用他们的平台和网络资源帮助我们提高活动的影响力,也更能吸引商家。
\end{enumerate}

\subsection{个人:}
通过科大校友群和各院系的系友群,以及代培生的有关群,找到一些成功的系友的联系方式。或者找到那段时间在合肥及其周边城市工作或出差的校友,通过宣传和动员,邀请他们回学校参加相关活动。希望其可以为本次的活动提供一定的经费支持。

%%%%%%%%%%%%%%%%%%%%%%%%%%%%%%%%%%%%%%%%%%%%%%%%%%%%%%%%%%%%%%%%%%%%%
\section{活动主题及意义;宣传部分}
\subsubsection{活动部分}
\begin{enumerate}
\item 趣味篮球赛
主题:趣味篮球赛,有你更精彩
目的:为了丰富代培生的课外娱乐生活,调动同学们参加体育锻炼的热情,同时能增强体质,增进同学们之间的友谊,给大家打造一个相互交流、相互认识学习的平台,让代培生同学们能更好的融入科大生活。

\item 羽毛球赛
主题:激情绽放,“羽”您共享
目的:为了丰富代培生同学们的课外娱乐生活,培养积极向上的进取精神。本着“友谊第一,比赛第二”的原则,我们策划组织这场羽毛球赛,不仅能提高同学们间的凝聚力,同时也倡导了一种热爱运动的氛围,我们期盼大家积极热情的参加。

\item 素质拓展
主题:沟通  友谊  团结  协作  
目的:1:通过游戏互动参与来引导受训的每一位同学走出以往封闭的自我,主动与人沟通,打开心扉以完全参与者的态度来亲身体验与察觉,认识自己的闪光点与不足,形成完整的自我概念,树立积极的生活态度。
2:安排了很多有意添加幽默的游戏,或为解决困难而采取的活动。使受训同学们在训练中产生愉快感,从而达到心理压力的释放或排解,在轻松、相互信赖的团队气氛中体验成长。
	
\end{enumerate}

\subsection{宣传部分}
\subsubsection{宣传途径:}
\begin{itemize}
	\item 海报宣传:由部内成员进行排版规划,出一期关于本次活动内容的海报,张贴在食堂图书馆教学楼实验楼等处,将电子版播放在宿舍楼和各学院办公实验楼下的电子屏幕上;
	费用:一张海报需10元,需20张海报,共计200元;
	人手:海报设计需1到2人,东西区张贴海报需6人。
	\item 拉横幅:在活动中心等地拉起宣传本活动的横幅;
	费用:一条横幅20元,需6条横幅,共计120元;
	人手:需6人。
	\item 发宣传单:打印不多不少的宣传单,在食堂教学楼等必经处分发进行宣传;
	费用:500元可印1500张左右
	人手:需16人参与。
	\item QQ群宣传:转发活动海报到各个研究生班级群里;
	线上宣传 无费用
	\item 微信微博宣传:校研会微信公共号和微博号推送活动的相关消息。
	线上宣传 无费用 
\end{itemize}

总计费用:需820元左右,人手不重复使用需30人左右。

\subsubsection{所需联系单位:}海报横幅传单制作单位;宿舍楼及各办公楼电子屏管理者。
%%%%%%%%%%%%%%%%%%%%%%%%%%%%%%%%%%%%%%%%%%%%%%%%%%%%%%%%%%%%%%%%%%%%%

\newpage
\section{协作分工}
\subsection{组长}
- 江金健
\subsection{策划书部分}
\begin{itemize}
	\item 篮球赛 - 靳亚雪
	\item 素质拓展 – 周倩芳
	\item 羽毛球赛 - 王怡然
	\item 整合– 张宇
\end{itemize}
\subsection{活动主题和意义;宣传部分}
- 苏鹏
\subsection{外联及赞助部分}
- 王叶竹
\subsection{PPT制作和答辩}
- 许仕杰









	\end{CJK}
\end{document}